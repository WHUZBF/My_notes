% Copyright 2015 by Kpym
%
% This file may be distributed and/or modified
%
% 1. under the LaTeX Project Public License and/or
% 2. under the GNU Public License.
%
\typeout{tikz-avoidpath v1.0}

% some additional libraries used by this one
\usetikzlibrary{calc,math,decorations.markings,patterns}

% style for arrows on the path
% Usage :
%  [with arrows=35mm] put arrows every 35mm
%  [with arrows at positions={35mm,70mm,140mm}] put arrows at 35mm,70mm,140mm
% -----------------------------------------------------------
\tikzset{
  with arrows/.style={
    decoration={ markings,
      mark=between positions #1 and .999 step #1 with {\arrow{stealth}}
    }, postaction={decorate}
  }, with arrows/.default=13mm,
  with arrow at/.style={
    decoration={ markings,
      mark=at position #1 with {\arrow{stealth}}
    }, postaction={decorate}
  }, with arrows/.default=13mm,
  with arrows at positions/.style={
   with arrow at/.list={#1}
  }
}

% style to draw reversed circle (helpfull when filling, to not fill inside)
% Usage :
%   circle[reversed with radius=1cm]
% -----------------------------------------------------------
\tikzset{
  reversed with radius/.style={
    x radius=#1,
    y radius=-#1,
  }
}

% macros to use tikzmath inside 'to path'
% -----------------------------------------------------------
\newcommand\tikzmathextra[1]{\pgfextra{\tikzmath{#1}}}
\newcommand\insertpath[1]{\pgfkeys{/tikz/insert path={#1}}}

% macros for data dependancies
% Usage :
%  .pgfmathparse in=\macroname is like ".store in", but mathparse it first
%  .pgfmathsync to=\macroname{<formula>} every time the key is set, the <formula> is mathparsed and stored in \macroname
% -----------------------------------------------------------
\pgfkeys{
  /handlers/.pgfmathparse in/.code=\pgfkeysalso{\pgfkeyscurrentpath/.code=\pgfmathsetmacro#1{##1}\ifpgfmathunitsdeclared\edef#1{#1 pt}\else\edef#1{#1 cm}\fi},
  /handlers/.pgfmathsync to/.code 2 args=\pgfkeysalso{\pgfkeyscurrentpath/.append code=\pgfmathparse{#2}\edef\result{\pgfmathresult}\ifpgfmathunitsdeclared\providecommand#1{\result pt}\else\providecommand#1{\result cm}\fi}
}

% precision : pgfmathparse is not very precise we consides 0 <=> < \avoidpointprecision
%  in particular two point are considered identical if the distance is < \avoidpointprecision
%  for the same reason \avoidonaxeprecision is used to check if the pole is on the line (\tikztostart)--(\tikztotarget)
% -----------------------------------------------------------
\tikzset{
  /avoid/.cd,
  point precision/.store in=\avoidpointprecision,
  on axe precision/.store in=\avoidonaxeprecision,
  precision/.style={point precision=#1,on axe precision=#1},
  precision=.7pt % the initial precision
}

% main to path
% -----------------------------------------------------------
\tikzset{
  % styles to set the variables
  /avoid/.cd,
    pole/.store in=\avoidpole,
    radius/.pgfmathparse in=\avoidradius,
    radius/.pgfmathsync to=\avoidneck{\avoidradius},
    neck/.pgfmathparse in=\avoidneck,
    neck/.pgfmathsync to=\avoidradius{\avoidneck},
    side/.store in=\avoidside, side=0,
    left/.style={side=1},
    right/.style={side=-1},
  /tikz/.cd,
  % the main style
  avoid/.style={/avoid/.cd,#1,/tikz/.cd, % set the varaibles
    to path={
      \tikzmathextra{
        point \p; % points
        real \a; % angles (in degree)
        \p{start} = (\tikztostart);
        \p{target} = (\tikztotarget);
        \p{pole} = \avoidpole; % \avoidpole is like (A) or (1,0)
        \p{projection} = (\p{start})!(\p{pole})!(\p{target});
        if (abs(\px{projection}-\px{pole}) + abs(\py{projection}-\py{pole})) < \avoidonaxeprecision then {
          % if the pole is on the main axe (left and right options are valid)
          \p{first base} = (\p{projection})!\avoidradius!(\p{start});
          \a{start} = atan2(\py{start}-\py{projection},\px{start}-\px{projection});
          if \avoidside >= 0 then {
            \a{target} = \a{start}-180; % left
          } else {
            \a{target} = \a{start}+180; % right
          };
          if (abs(\px{first base}-\px{start}) + abs(\py{first base}-\py{start})) > \avoidpointprecision then {
            {\insertpath{-- (\p{first base})}};
          };
          {\insertpath{arc(\a{start}:\a{target}:\avoidradius)}};
        } else {
          % if pole outside of the main axe (left and right options are useless)
          \sameasneck = abs(veclen(\px{target}-\px{start},\py{target}-\py{start})-\avoidneck) < \avoidpointprecision;
          if \sameasneck then {
            % if |\p{target}-\p{start}| ~= \avoidneck
            \p{first base} = (\p{start});
            \p{second base} = (\p{target});
          }
          else{
            \p{first base} = (\p{projection})!\avoidneck/2!(\p{start});
            \p{second base} = (\p{projection})!\avoidneck/2!(\p{target});
          };
          \p{neck vector} = (\p{pole})!sqrt((\avoidradius)^2-(\avoidneck/2)^2)!(\p{projection})-(\p{projection});
          \p{first on circle} = (\p{first base}) + (\p{neck vector});
          \p{second on circle} = (\p{second base}) + (\p{neck vector});
          \a{start} = atan2(\py{first on circle}-\py{pole},\px{first on circle}-\px{pole});
          \a{target} = atan2(\py{second on circle}-\py{pole},\px{second on circle}-\px{pole});
          if abs(\a{start}-\a{target}) < 180 then {
            if \a{start} < \a{target} then {
              \a{start} = \a{start} + 360;
            }
            else {
              \a{target} = \a{target} + 360;
            };
          };
          if not \sameasneck then {
            {\insertpath{-- (\p{first base})}};
          };
          {\insertpath{-- (\p{first base}) -- (\p{first on circle}) arc(\a{start}:\a{target}:\avoidradius)}};
          if not \sameasneck then {
            {\insertpath{-- (\p{second base})}};
          };
        };
      }% end tikzmathextra
      --(\tikztotarget) \tikztonodes
    }% end to path
  }% end avoid/.style
}




