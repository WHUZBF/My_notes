% 这是有限群附录

\chapter{Finit Group}
这篇附录是根据李新征老师群论讲义个教学录像整理而成的笔记, 大致只打算写完有限群, 对于高量学习绰绰有余。
\section{群的基本结构}
\subsection{群的定义和一些基本定理}
其实在接受前面对于线性空间的抽象定义之后也很好接受群的概念:
\begin{define}{群的定义}
    群是一个有着\textbf{乘法}$\circ$这一特殊结构的元素集合$G$:
    \begin{itemize}
    \item[$\bullet$] \textbf{封闭性}:$\forall g_1,g_2\in G\rightarrow g_1\circ g_2\in G$
    \item[$\bullet$] \textbf{结合律}:$(g_1\circ g_2)\circ g_3=g_1\circ (g_2\circ g_3)$
    \item[$\bullet$] \textbf{单位元(幺元)存在性}:$\exists e\in G,\mathrm{s.t.}\forall g\in G\rightarrow g\circ e=e\circ g=g$
    \item[$\bullet$] \textbf{逆元存在性}:$\forall g\in G,\exists g^{-1}\in G\rightarrow g\circ g^{-1}=g^{-1}\circ g=e$
    \end{itemize}
\end{define}
上面的定义中不必额外强调单位元和逆元的唯一性, 因为根据定义唯一性自动存在。而且有\footnote{后面在不引起歧义时, 省略乘法符号}:
\begin{equation*}
    (g_1g_2)^{-1}=g_2^{-1}g_1^{-1}
\end{equation*}

不过要注意, 一般的群的乘法是\textbf{不满足}交换律的, \textbf{乘法额外满足交换律的群我们称之为Abel群}。
\begin{example}{群的例子}
    全体整数$\mathbb{Z}$在自然加法作为乘法的情况下构成一个群, 而且还是一个Abel群。
\end{example}
\begin{define}{群的阶}
    群的阶的定义和集合的势的概念是相通的, 定义为群内元素的个数, 记为$|G|$, 常说的有限群指的就是群的阶有限大。
\end{define}
下面要介绍的定理是后面的证明几乎都需要用到的定理:
\begin{theorem}{重排定理}
    \begin{equation*}
        gG=Gg=G
    \end{equation*}
    其中$g\in G$, 其实就是说$G$中拿出一个元素再去和$G$中的所有元素相乘, 得到的集合还是原来的群$G$本身, 而且不会出现重复。$g$的作用只是把原先的群元素重新排列了一下。
\end{theorem}
\begin{proof}
     $(1)\forall g_\alpha \in G$,由逆元存在性和封闭性知$g^{-1}g_\alpha \in G$,那么再根据结合律和逆元定义知$g(g^{-1}g_\alpha)=g_\alpha$,那么当
    $g_\alpha$取遍$G$中元素我们便得到了$gG=G$.\\
    $(2)$用到群相关定理证明的大杀器:{\color{red}{反证法}},设$g_\alpha\neq g_\beta$,如果$gg_\alpha=gg_\beta$,那么$g^{-1}(gg_\alpha)=g^{-1}(gg_\beta)\Rightarrow g_\alpha= g_\beta$,
    与已知矛盾.这一证明其实用掉了群的所有性质.\qed
\end{proof}
\subsection{群的内部结构}
\begin{define}{子群}
    设$H$是群$G$的一个子集, 如果$H$中的元素在$G$的乘法定义下也构成一个群, 那么就称$H$是$G$的一个子群, 记为$H\leq G$
\end{define}
\begin{example}{平庸子群}
    显然任何群$G$的幺元和其本身是$G$的两个子群,我们称之为平庸子群,其它的我们称为$G$的固有子群。
\end{example}
\begin{define}{循环子群和群元的阶}
    对任意一个\textbf{有限群}$G$,从中任取一个元素$a$,在原先的群乘法定义下作幂操作,\uwave{总是}可以得到一个$G$的子群$Z_k\equiv{a,a^2,a^3,\ldots,a^k=e}$,我们称之为
    $G$的一个循环子群,这个子群的阶,或者说使得$a^k=e$的最小的$k$称之为群元$a$的阶.
\end{define}
这里需要额外说明的就是为啥任何一个元素一定可以生成一个循环子群。只需要说明使$a^k=e$的$k$一定存在,$a=e$,问题解决;$a\neq e$,那么$a^2\neq a$否则$a=e$,$a^2=e$那么问题解决,
但如果$a^2\neq e$,就再考虑$a^3$,如此下去,由于前提是$G$为有限群,所以这个$k$一定存在。
\begin{define}{陪集}
    设$H\leq G$,由固定的$g\in G$可以生成$H$的左陪集:$gH\equiv{gh|h\in H}$,或是右陪集:$Hg\equiv{hg|h\in H}$
\end{define}
注意上面的定义中我们并没有用“群”这个字眼,说明陪集这个玩意一般只是一个集合而已,没有群结构。不难证明陪集中的元素是和子集$H$中的元素一一对应的,根据重排定理,
$g\in H$时$gH$和$Hg$构成一个群,就是$H$本身。
\begin{theorem}{陪集定理}
    $H\leq G, g_1,g_2\in G\rightarrow \{g_1H=g_2H\}\oplus \{g_1H \cap g_2H=\varnothing\}$\footnote[0]{其中$\oplus$表示“异或”,表示两者只能取其一。}
\end{theorem}
这个定理使用重排定理和反证法很好证明,只需要说明只要两个左陪集有一个公共元素那么$g_1H=g_2H$即可。当然,这个定理对于右陪集一样适用。这个定理也告诉我们或许任意一个群可以分解为一系列其某个子群的陪集的不交并。
\begin{proposition}{陪集分解}
    $H\leq G$,则$G$一定可以分解为一系列陪集的不交并,及:
    \begin{equation*}
        G=eH+g_1H+g_2H+\cdots
    \end{equation*}
\end{proposition}
根据前面的陪集定理,我们只需要在构建这个分解时,不断选取$g_\alpha$不属于前面的集合去构建新的陪集$g_\alpha H$即可。前面我们说过陪集元素和$H$一一对应,那么$|gH|=|H|$,再看陪集分解,由
于我们可以将任意一个集合分解为一系列陪集的\textbf{不交并},所以我们立即得到下面的Lagange定理:
\begin{theorem}{Lagrange定理}
    有限子群的阶必为群的阶的因子,也就是说$|H|$一定可以整除$|G|$
\end{theorem}
\begin{define}{共轭}
    $\forall f,h\in G$,如果$\exists g\in G$,使得$gfg^{-1}=h$,我们就称这两个元素共轭,记为$f\sim h$
\end{define}
群元素共轭的概念抽象一点这就是集合中的\textbf{等价关系的概念},满足下面三条:
\begin{itemize}
    \item[$\bullet$] \textbf{对称性}:$f\sim h\rightarrow h\sim f$
    \item[$\bullet$] \textbf{传递性}:$f_1\sim h,h\sim f_2\rightarrow f_1\sim f_2$
    \item[$\bullet$] \textbf{反身性}:$f\sim f$
\end{itemize}
上面的三条性质共轭全部满足,说的具体一点所有的$n\times n$矩阵构成一个群(这其实是个无限群),共轭的概念就是矩阵之间的相似概念。
\begin{define}{类}
    群$G$可按共轭关系分割成一些等价类$A_a={gag^{-1}|\forall g\in G}$.称$A_a$为群$G$的元素$a$的共轭类,简称为群的$a$类.
\end{define}
共轭类有下面的性质:
\begin{itemize}
    \item (1)单位元素$e$自称一类;
    \item (2)Abel群中的所有元素都自成一类;
    \item (3)类中的所有元素的阶都相等。
\end{itemize}

由于两个共轭类之间也有类似陪集之间的不相交的性质,所以我们也很容易实现将一个群用其群元的类来分解,只要对每个元素确定其类即可。不过前面的陪集分解是分解为一系列
大小相等的等份,而按照类的分解就不一定是等分了,不过还是有下面类似的定理:
\begin{theorem}{类中的元素个数}
    有限群的每个元素确定的类中的元素的个数都是群的阶的因子
\end{theorem}
\begin{proof}
    我们的目的是找到群的任意元素$g$确定的类的元素个数。

    证明这个定理的第一步是证明$\forall g\in G$,定义$H_g\equiv\{h\in G|gh=hg\}$,也就是所有与$G$中元素互易的元素构成了$G$的一个子群。由于子群乘法的定义来源于$G$,是良好定义的,
    所以我们只需要证明封闭性和逆元存在性。

    先证封闭:$gh_1=h_1g,gh_2=h_2g\rightarrow gh_1h_2=h_1gh_2=h_1h_2g$

    再证有逆:$gh=hg\rightarrow h^{-1}g^{-1}=g^{-1}h^{-1}\rightarrow gh^{-1}=h^{-1}g\rightarrow h^{-1}\in H_g$

    第二步是将$G$陪集分解为$\{g_0H_g,g_1H_g,\ldots\}$,其中$g_0$是幺元。注意到每个陪集$g_iH_g$中,$h_\alpha$取遍$H_g$,也就是$g_i h_\alpha$取遍$g_iH_g$中的所有元素时,
    $(g_i h_\alpha )g(g_i h_\alpha )^{-1}$给出的是和$g$共轭的元素这是毋庸置疑的,我们要是能进一步证明其实给出的还是同一个元素,我们记为$\tilde{g_i}$,我们就把陪集$g_iH_g$和$g$类中的元素
    对应起来了,如果不同的陪集给出不同的元素,也就说还可证明这种对应是一一对应的。那么根据Lagrange定理,陪集$g_ih_g$将$G$等分,而$g$类中元素个数就是等分的份数,也就是$|G|$的因子。下面着手证明:

    $(1)$一个陪集对应一个$g$类中元素:$(g_i h_\alpha )g(g_i h_\alpha )^{-1}=(g_i h_\alpha )g( h_\alpha^{-1}g_i^{-1} )=g_i (h_\alpha g) h_\alpha^{-1}g_i^{-1}
    =g_i  g(h_\alpha h_\alpha^{-1})g_i^{-1} =\tilde{g_i}$
    
    $(2)$不同陪集给出不同元素:假设$g_iH_g\neq g_j H_g$,那么如果对应的元素相等:$g_igg_i^{-1}=g_jgg_j^{-1}$,那么$(g_j^{-1}g_i)g=g(g_j^{-1}g_i)\rightarrow g_j^{-1}g_i\in H_g$,
    根据重排定理$g_j^{-1}g_iH_g=H_g\rightarrow g_iH_g=g_jH_g$,与假设矛盾。\qed
\end{proof}
\begin{define}{共轭子群}
    $H\leq G,K\leq G$,如果$\exists g\in G\rightarrow gHg^-1\equiv\{ghg^{-1}|h\in H\}=K$
\end{define}
这个定义完全是照搬群元共轭的,后面用到的机会也不多。
\begin{define}{正规子群(不变子群)}
    如果子群$H$中的所有元素的同类元素都属于$H$,那么称$H$是$G$的不变子群或者说正规子群,在群元共轭操作下是不变的,记为$H\unlhd G$.
\end{define}
下面这个定理是和前面的定义等价的,有的书常常也用这个定理作为正规子群定义。
\begin{theorem}{正规子群左右陪集完全重合}
    $H\unlhd G\Leftrightarrow \forall g\in G,gH=Hg$
\end{theorem}
\begin{proof}
    先证必要性,等价于证明$H\unlhd G\rightarrow gHg^{-1}=H$.

    根据正规子群的定义,$\forall h_\alpha\in H\rightarrow gh_\alpha g^{-1}\in H\rightarrow gHg^{-1}\subseteq H$,反过来,还是根据正规子群的定义以及$g^{-1}\in G$,
    $\forall h_\alpha\in H$,一定存在$h_\beta\in H$,使得$g^{-1}h_\alpha (g^{-1})^{-1}=h_\beta$成立,而这意味着$h_\alpha=g h_\beta g^{-1}\in gHg^{-1}$,也就是说$gHg^{-1}\supseteq H$.这样我们便证明了必要性.

    再证充分性,$\forall h_\alpha\in H$,根据左右陪集相等,一定存在$h_\beta\in H$,使得$gh_\alpha g^{-1}=h_\beta$,而$g$是在$G$中任取的,这也便直接说明了$H$中元素的类都属于$H$.
    \qed
\end{proof}
对于正规子群,今后不用再区分左右陪集。
\begin{theorem}{两个陪集中元素的乘积必为第三个陪集中的元素}
    $H\unlhd G$,取$g_1H\neq g_2H$,且两陪集都不是子群本身,那么$\forall g_1h_\alpha\in g_1H,g_2h_\beta\in g_2H\rightarrow g_1h_\alpha g_2h_\beta\in g_3H$,其中$g_3H\neq g_1H,g_3H\neq g_2H$
\end{theorem}
\begin{proof}
    如果$g_1H=g_2H=g_0H$,两个陪集都是子群本身,那么显然得到的元素都是$H$中的元素;

    如果$g_1H,g_2H$中有一个是$g_0H$,那么显然最后得到的元素要么是$g_1H$中元素,要么是$g_2H$中元素。

    现在考虑$g_1H$和$g_2H$都不是$H$的情况,$g_1h_\alpha g_2 h_\beta=g_1g_2(g_2^{-1}h_\alpha g_2 )h_\beta$,根据正规子群定义,$g_2^{-1}h_\alpha g_2\equiv h_{\alpha^\prime}\in H$,
    如果$g_1h_{\alpha} g_2 h_\beta\in g_1H$,那么$\exists h_\gamma\in H$,使得$g_1h_\gamma=g_1g_2h_{\alpha^\prime} h_\beta\rightarrow g_2=h_\gamma h_\beta^{-1} h_{\alpha^\prime}^{-1}\in H$,根据重排定理这直接说明$g_2 H=H$.与假设矛盾,关于$g_1$的情形是对称的
    ,证明略去. 

    而且这个证明还让我们了解到这个新的陪集是$g_1g_2H$,所以两个陪集中的元素相乘给出的新的元素都来自于同一个新的陪集。
    
    你还可以简单的接纳下面这个非常\uwave{物理仁}的证明:\[(g_1H)(g_2H)=g_1(Hg_2)H=g_1(g_2H)H=g_1g_2(HH)=g_1g_2H\]
    
    \qed
\end{proof}
\begin{define}{商群}
    $H\unlhd G$,用这个不变子群对$G$进行陪集分解:$\{g_0H,g_1H,g_2H,\ldots,g_nH\}$.把其中的每一个陪集当作是一个新的元素,记$f_i=g_iH$.
    对这个陪集的集合定义新的乘法,$f_if_j=f_k$表示陪集$g_iH$和$g_jH$中的元素相乘得到$g_kH$中的元素.\textbf{那么这个陪集的集合在这个乘法的定义下构成一个群,称作$G$的一个商群},记为$G/H$.
\end{define}
这个概念其实有点像线性空间里面的仿射子集建立的商空间。
\subsection{同构与同态}